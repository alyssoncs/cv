\documentclass[]{../document-class/twentysecondcv}

\usepackage[T1]{fontenc}
\usepackage[utf8]{inputenc}

\begin{document}

\profilepic{../res/alysson.jpeg}
\cvname{Alysson Cirilo}
\cvjobtitle{Engenheiro de Software}
\cvaddress{Brasil}
\cvmail{alysson.cirilo@gmail.com}
\cvsite{https://www.github.com/alyssoncs}

\aboutme{
	Eu sempre quis aprender como as coisas realmente funcionam por baixo dos panos, e desde que eu recebi meu primeiro computador tornou-se claro para mim o que eu faria pelo resto dos meus dias. Eu acredito fortemente que criar tecnologia é uma das atividades mais humanas que existem.
}

\skills{
	{Português/5},
	{Inglês/4.5},
	{C Programming Language/4.5},
	{Java/4.2},
	{Kotlin/4.0},
	{Linux/4.0},
	{Android Development/4},
	{Arduino/3.7},
	{Git/3.5},
	{Docker/3.5},
	{MongoDB/2.0}}

\makeprofile

\section{interesses}

Eu me tornei muito interessado em assuntos como algoritmos, sistemas operacionais, desenvolvimento de sistemas, IoT e muito mais.

Desenvolvimento de sistemas, desenvolvimento Android e arquitetura de software são alguns dos assuntos que eu estou atualmente interessado e estudando.

\section{educação}

\begin{twenty}
	\twentyitem
		{2013--2019}
		{B.Sc. em Ciência da Computação}
		{UFMA}
		{\emph{Descoberta e Desconexão de Objetos Inteligentes em Ambientes Oportunísticos de IoMT}}
\end{twenty}

\section{experiência}

\begin{twenty}
	\twentyitem
		{2020--Atual}
		{Desenvolvedor Pleno}
		{Secretaria de Estado de Administração Penitenciária do \\\hspace*{\fill}Maranhão (SEAP)}
		{
			$\bullet$ Desenvolvimento da versão mobile Android com Kotlin do sistema que permite a busca de informações dos detentos e foragidos do estado.
		}

	\twentyitem
		{2019--2020}
		{Desenvolvedor Júnior}
		{Secretaria de Estado de Administração Penitenciária do \\\hspace*{\fill}Maranhão (SEAP)}
		{
			$\bullet$ Me tornei responsável pelo projeto, construção e implantação de um serviço e API flexível para armazenar informação de \textit{logs} de outros sistemas utilizando Spring Boot e MongoDB;\vskip 4pt 

			$\bullet$  Implementação de aplicação móvel seguindo os princípios do Clean Architecture, utilizando Android nativo e Firebase Cloud Messaging, esse app permitiu que visitantes agendassem visitas virtuais para seus familiares presos, recurso importante durante a pandemia do Covid-19;\vskip 4pt

			$\bullet$ Também parte do grupo responsável pela manutenção e implementação de novas funcionalidades do sistema que gerencia todas as informações prisionais do estado.
		}
	
	\twentyitem
		{2019--2019}
		{Estagiário}
		{Laboratório de Sistemas Distribuídos Inteligentes \\\hspace*{\fill}(LSDi)}
		{
			Desenvolvimento de sistema para localização \textit{indoor} baseado em \textit{beacons bluetooth}, consistindo em:\vskip 4pt
          
			$\bullet$ Desenvolvimento de sistema Web que armazena todos os relacionamentos entre usuários, \textit{beacons} bluetooth e espaços físicos usando o framework Spring Boot e PostgreSQL;\vskip 4pt
        
			$\bullet$ Desenvolvimento de aplicação Android capaz de detectar os \textit{beacons} no ambiente e registrar este encontro em um serviço Web utilizando Retrofit.
		}
			
	\twentyitem
		{2017--2019}
		{Aluno de Iniciação Científica}
		{Laboratório de Sistemas Distribuídos Inteligentes \\\hspace*{\fill}(LSDi)}
		{
			$\bullet$ Implementação de novas funcionalidades em um middleware de IoT existente usando Java na plataforma Android, incluindo o mecanismo de notificação para descoberta, conexão e desconexão de objetos inteligentes;\vskip 4pt

			$\bullet$ Desenvolvimento de software embarcado usando a linguagem de programação C em placas Arduino, incluindo o uso de tecnologia \textit{Bluetooth Low Energy};\vskip 4pt

			$\bullet$ Integração entre software embarcado e móvel usando as tecnologias \textit{Bluetooth Low Energy} e MQTT.
		}
\end{twenty}

\section{outras informações}

Buscar uma formação em ciência da computação foi o próximo passo lógico a ser feito. Eu me tornei apaixonado por diversos campos.

Eventualmente eu cheguei a conclusão que o domínio dos conceitos básicos, os fundamentos, é o caminho a se seguir para se tornar um profissional competente, então eu geralmente sigo uma abordagem \textit{bottom-up} ao estudar qualquer tópico.

\end{document} 

