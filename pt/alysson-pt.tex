\documentclass[]{../document-class/twentysecondcv}

\usepackage[T1]{fontenc}
\usepackage[utf8]{inputenc}

\begin{document}

\profilepic{../res/alysson.jpeg}
\cvname{Alysson Cirilo}
\cvjobtitle{Engenheiro de Software}
\cvaddress{Brasil}
\cvmail{alysson.cirilo@gmail.com}
\cvsite{https://www.github.com/alyssoncs}

\aboutme{
	Eu sempre quis aprender como as coisas realmente funcionam por baixo dos panos, e desde que eu recebi meu primeiro computador tornou-se claro para mim o que eu faria pelo resto dos meus dias. Eu acredito fortemente que criar tecnologia é uma das atividades mais humanas que existem.
}

\skills{
	{Português/5},
	{Inglês/4.5},
	{C Programming Language/4.5},
	{Java/4.2},
	{Kotlin/4.0},
	{Teste de Software/4.0},
	{Android Development/4.0}}

\makeprofile

\section{interesses}

Desenvolvimento Android, arquitetura de software e teste de software são alguns dos assuntos que eu estou atualmente interessado e estudando.

\section{educação}

\begin{twenty}
	\twentyitem
		{2013--2019}
		{B.Sc. em Ciência da Computação}
		{UFMA}
		{\emph{Descoberta e Desconexão de Objetos Inteligentes em Ambientes Oportunísticos de IoMT}}
\end{twenty}

\section{experiência}

\begin{twenty}
	\twentyitem
		{2021--Atual}
		{Desenvolvedor Android}
		{iFood}
		{
			Trabalhou na equipe de checkout com uma arquitetura baseada em plugins.
		}

	\twentyitem
		{2020--2021}
		{Desenvolvedor Android}
		{Meta}
		{
			Trabalhou com a equipe de desenvolvimento da 4all, construindo um app de carteira digital e pontos de cartão de crédito para um grande banco. \vskip 4pt

			$\bullet$ Integração da ferramenta de cobertura de código em um projeto Android multi-módulo; \vskip 4pt

			$\bullet$ Implementação do subsistema responsável pela assinatura digital de todos as requests http feitas pelo app; \vskip 4pt

			$\bullet$ Design de uma abstração de autenticação biométrica, possibilitando a escrita de testes unitários em fluxos críticos da aplicação.
		}

	\twentyitem
		{2020--2020}
		{Desenvolvedor Pleno}
		{Secretaria de Estado de Administração Penitenciária do \\\hspace*{\fill}Maranhão (SEAP)}
		{
			$\bullet$ Aplicativo Android que permite a busca de informações dos detentos e foragidos do estado.
		}

	\twentyitem
		{2019--2020}
		{Desenvolvedor Júnior}
		{Secretaria de Estado de Administração Penitenciária do \\\hspace*{\fill}Maranhão (SEAP)}
		{
			$\bullet$ Projeto, construção e implantação de uma API flexível para armazenar informação de \textit{logs} de outros sistemas utilizando Spring Boot e MongoDB;\vskip 4pt 

			$\bullet$ Aplicativo Android seguindo Clean Architecture, utilizando Firebase Cloud Messaging, permitiu o agendamento de visitas virtuais com pessoas presas, recurso importante durante a pandemia do Covid-19;\vskip 4pt

			$\bullet$ Participação na equipe responsável pelo sistema que gerencia todas as informações prisionais do estado.
		}
	
	\twentyitem
		{2019--2019}
		{Estagiário}
		{Laboratório de Sistemas Distribuídos Inteligentes \\\hspace*{\fill}(LSDi)}
		{
			Desenvolvimento de sistema para localização \textit{indoor} baseado em \textit{beacons bluetooth} consistindo em:\vskip 4pt
          
			$\bullet$ Sistema Web que gerencia o relacionamento entre os usuários, \textit{beacons} e espaços físicos usando Spring Boot e PostgreSQL;\vskip 4pt
        
			$\bullet$ Aplicação Android capaz de detectar os \textit{beacons} no ambiente e registrar este encontro utilizando Retrofit.
		}
			
	\twentyitem
		{2017--2019}
		{Aluno de Iniciação Científica}
		{Laboratório de Sistemas Distribuídos Inteligentes \\\hspace*{\fill}(LSDi)}
		{
			$\bullet$ Implementação de mecanismo de notificação para descoberta, conexão e desconexão de objetos inteligentes em middleware de IoT Android;\vskip 4pt

			$\bullet$ Desenvolvimento de software embarcado para Arduino, incluindo o uso de tecnologia \textit{Bluetooth Low Energy};\vskip 4pt

			$\bullet$ Integração entre software embarcado e móvel usando as tecnologias \textit{Bluetooth Low Energy} e MQTT.
		}
\end{twenty}

\end{document} 

